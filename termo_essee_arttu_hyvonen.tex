\documentclass[12pt,a4paper,titlepage]{article}
\usepackage[utf8]{inputenc}
\usepackage[finnish]{babel}
\usepackage{setspace}
\usepackage{parskip}
\usepackage{amssymb}
\usepackage{amsmath}
\usepackage{graphicx}
\usepackage{fancyhdr}
\usepackage[top=1in, bottom=1in, left=1in, right=1in]{geometry}
\usepackage{float}
\usepackage[shellescape]{gmp}
\usepackage[section]{placeins}
%\usepackage[numbered,autolinebreaks,useliterate]{mcode} % jos tahdot laittaa matlabkoodia näkyville niin kannattaa käyttää tätä

% hyödyllisiä paketteja:
\usepackage{siunitx}\sisetup{per=frac} % SI-yksiköitä.
%\usepackage{supertabular} % jos tarttee isoja taulukoita
%\usepackage{fullpage} % pienemmät marginaalit jos haluaa

\usepackage{listings}
\usepackage{color}

\definecolor{dkgreen}{rgb}{0,0.6,0}
\definecolor{gray}{rgb}{0.5,0.5,0.5}
\definecolor{mauve}{rgb}{0.58,0,0.82}

\lstset{frame=tb,
  language=python,
  aboveskip=1mm,
  belowskip=1mm,
  showstringspaces=false,
  columns=flexible,
  basicstyle={\small\ttfamily},
  numbers=none,
  numberstyle=\tiny\color{gray},
  keywordstyle=\color{blue},
  commentstyle=\color{dkgreen},
  stringstyle=\color{mauve},
  breaklines=true,
  tabsize=3
}


\usepackage{hyperref} % lisääthän omat pakettisi ENNEN hyperref'iä
\hypersetup{pdfborder={0 0 0}}
\onehalfspacing
\cfoot{}
\rhead{\thepage}
% asettaa nyk. kappaleen nimen vasempaan ylänurkkaan, saa poistaa jos haluaa
\lhead{\leftmark}

%%%%% kaikki ennen tätä liittyy käytettäviin paketteihin tai dokumentin muotoiluun. siihen ei tarvinne aluksi koskea. %%%%%

%%%%% kansilehti %%%%%
\title{Termo essee}
\author{\begin{tabular}{c}
Arttu Hyv{\"o}nen, 014808984
\end{tabular}}
\date{\today}
\begin{document}
\maketitle
\newgeometry{top=1in, bottom=1.5in, left=1.5in, right=1.5in}

\restoregeometry

% Sisällysluettelo
\thispagestyle{empty}
\tableofcontents
\newpage
\setcounter{page}{1}
\parskip=1em \advance\parskip by 0pt plus 2pt
\pagestyle{fancy}

% prosenttimerkillä alkavat rivit ovat kommentteja: niitä ei katsota dokumenttia käännettäessä eli ne ovat vain kirjoittajaa varten

%%%%%%%%%%%%%%% Oleellinen sisältö alkaa%%%%%%%%%%%%%%%
\section{Johdanto}

\newpage
\section{Osa1}

\begin{figure}[H]
\centering
%\includegraphics[width=\textwidth]{linis2_harjoitustyo_1.png}
\caption{Telttafunktioiden kuvaajat välillä $[0,1]$}
\end{figure} 

\newpage
\section{Johtopäätökset}


\newpage
\section{Lähdeluettelo}


%%%%% Sisältö loppuu, lähdeluettelo %%%%%
\bibliographystyle{plain}
\bibliography{selkkarilahteet} %lähdeluettelon tiedot tiedostossa selkkarilahteet.bib. Esimerkiksi helkasta saa kirjojen tiedot valmiiksi bibtex-muodossa, kannattaa hyödyntää.

\appendix

\end{document}